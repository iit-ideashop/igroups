\documentclass{article}
\usepackage{url}
\usepackage{hyperref}
\title{Using the IPRO Peer Review System\\For Faculty}
\author{Casey Bennett}

\begin{document}
\maketitle
\tableofcontents
\newpage

\section{Introduction}

The IPRO Peer Review System is an online application written in PHP that allows IPRO students to provide fellow students with feedback on their performance over the course of a semester. You, the IPRO instructor, can also view this feedback to assist with your grading procedure, to identify problems with your team that you may be able to correct, and to identify outstanding students in your team.

\section{Getting Started}

To log into the Peer Review System, visit \url{http://sloth.iit.edu/~iproadmin/peerreview}. Your username and password are identical to those used for iGroups. If you are unable to log in, you should first try resetting your password using the "Reset Password" link located under the login form. If resetting your password does not enable you to log in, contact \url{mailto:iproadmin@iit.edu}.

\section{Group Status}

To view the status of your groups, select "View Status" from the left. Then, select the group whose status you would like to see.

You will see a list of all the people in your group, along with how many surveys they have completed. If the number of completed surveys is equal to the number of available surveys for this person, their row will have a yellow background. If the number of completed surveys is less than the number of available surveys, their row will have a light gray background.

You can send students emails using this interface. To do so, check the "Send Message" box for that student, then press the "Compose Message" button. At the next screen, compose the subject and body of the message you want to send, then press the "Send" button. Using this feature, you can inform your students that the Peer Review system is available for them to use, or remind students to complete the surveys.

\section{Managing Your Groups}

To manage your groups, select "Manage Groups" from the left. Then, select the group you would like to manage.

You will see two list of all the people in your group, one of faculty and one of students. Each person has a radio button next to their name, through which you can remove people from the group. If a student dropped the course and should not be in the Peer Review system for your group, you can remove them by selecting them and clicking "Remove User from Group". If there is a student who should not be reviewed, but should have faculty access to the group, you can use the "Designate Faculty" button to set them as faculty. Finally, if a member of the group is missing, you can add them using the "Add a New User to this Group" form.

Please be careful when designating a student as faculty. Only members of IIT faculty should be granted this status.

Finally, there is a button labeled "Save and Reset Data". This button removes all completed surveys from your group and resets the process anew. Use this button for a second run of the Peer Review system, for example to see how things have changed from the midterm to the final. Only use this feature once students have received their individual reports, and you have downloaded all reports for the group, as the reports will be inaccessible once a new run has been created.

\section{Managing Your Survey Items}

To manage your survey items, select "Manage Survey Items" from the left. Then, select the group for which you would like to manage survey items.

You will see a list of all the survey items your group has. At first, the only ones that will be listed are those set by the IPRO Administrator. These survey items cannot be modified or removed in any way, but are shown here for reference. Below that list is a form that allows you to add a survey item to this group, known here as a custom criterion. Add a name and description, and press the "Create Criterion" button to add the criterion to your group's survey.

Once you have added a custom criterion, your list of survey items changes. The custom criteria are shown in editable boxes, where you can change the criteria's names and descriptions, with a checkbox in the far right column that can be checked to delete that criterion. Once you have made your changes, click "Edit Criteria" to save them.

\section{Reports}

To view reports, select "Reports" from the left.

There are two types of reports in the Peer Review System. Team Reports are summaries of the ratings each team member received. You can preview team reports by selecting your group then clicking "Preview in Browser". You can also download them in a spreadsheet format by clicking "Download Report". Team Reports are intended for use by faculty only.

The second type of report is the Individual Report. This is a more detailed report generated for each student in a group. To see the Individual Reports, you must first select your group. Then, a drop down menu of the members of the team is displayed. Select a member and click "Preview in Browser" to preview the report, or "Download Report" to download it in a spreadsheet format.

Individual Reports are intended to be distributed to the students, and the Peer Review System makes it easy to do this. Once the Peer Review process is completed, you can distribute individual reports to your students by selecting your group and clicking "Distribute Reports". Emails will be sent to each student, with the spreadsheet version of their report attached. This should be done only once per run.

If you would prefer to distribute the Individual Reports yourself, you can download all individual reports in one zip file for your convenience. To do so, select your group and click "Download Reports".

\section{Getting Help}

If you encounter a problem with the software, please email the IPRO Administrator at \url{mailto:iproadmin@iit.edu}.

\end{document}
